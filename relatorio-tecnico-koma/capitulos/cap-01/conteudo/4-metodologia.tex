\subsection{Metodologia}

O projeto de gerenciamento financeiro adotou uma abordagem metodológica baseada no design thinking para a coleta de dados. Este processo incluiu extensas pesquisas online, entrevistas detalhadas com possíveis usuários da plataforma e uma análise minuciosa de aplicativos similares já existentes. A análise de dados concentrou-se principalmente nas informações obtidas durante as entrevistas, as quais foram cruciais para a definição e desenvolvimento do projeto.

O desenvolvimento da plataforma escolheu a interface web como principal ponto de interação, armazenando todas as informações essenciais em um banco de dados \textit{MariaDB}. A implementação técnica envolveu o uso de \textit{React.js (React)} para o frontend, \textit{Node.js (Node)} para o backend, e \textit{Prisma.js} para a modelagem.

A validação do desenvolvimento foi conduzida pelos próprios entrevistados, que desempenharam um papel fundamental ao fornecer feedback contínuo. Modificações e ajustes foram implementados conforme necessário, sem restrições temporais rigorosas.

Todo o ciclo de desenvolvimento, desde a concepção até a implementação, foi realizado exclusivamente pelo autor do projeto. Os recursos utilizados incluíram um notebook, um gravador e ferramentas de modelagem de diagramas como o draw.io.

Em relação às limitações enfrentadas, o projeto adotou uma estratégia de mitigação cortando funcionalidades consideradas não essenciais para o funcionamento fundamental do software.